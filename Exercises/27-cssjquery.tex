\Exercise{jQuery und CSS}
%
\par Erstellen Sie eine einfache Webseite mit der man durch die Eingabeelemente
Wert (größer $0$, \jvar{number}), Text (String, \jvar{text}), Anfügen (Ja /
Nein, \jvar{checkbox}) und Farbe (String, \jvar{color}) einfache
Balkendiagramme aufbauen kann. Außerdem sollte zusätzlich ein Button
(``Einfügen'') existieren.
%
\begin{figure}[!h]
\centering
    \begin{tikzpicture}
        \newlength{\myheight}
        \newlength{\mywidth}
        \definecolor{myred}{RGB}{192,80,77}
        \definecolor{myorange}{RGB}{247,150,70}
        \definecolor{myblue}{RGB}{79,129,189}
        
        % % %Linker Text
            \node at (-3.2,3) [right,text width=3cm] {\small
                Höhe $h_a$ von {\tt div.append}:
                    \[
                    H = h_a \times h_i / h,
                    \]
                $H$ ist echte Höhe von {\tt div.bar},
                $h_i$ die eingegebene Höhe und
                $h$ die eingegebene Höhe von {\tt div.bar}.
                };
            
        % % %Erster Balken
            \setlength{\myheight}{6.05cm}
            \setlength{\mywidth}{1.875cm}
            %Grundbalken
            \draw [dotted] (0,0) rectangle (\mywidth,\myheight);
            
            %Farbige Balken
            \draw [fill,myred] (0,0) rectangle (\mywidth,0.45 * \myheight);
            \draw [fill,myorange] (0,.45 * \myheight) rectangle (\mywidth,\myheight*.7);
            
            %Labels
            \node at (\mywidth / 2, \myheight - .75cm) {\bfseries Text};
            \node at (\mywidth / 2, \myheight*.575 + .25cm) [white] {\bfseries Text};
            \node at (\mywidth / 2, \myheight*.575 - .25cm) [white] {\footnotesize \tt
            div.append};
            \node at (\mywidth / 2, \myheight*.2125 + .25cm) [white] {\bfseries Text};
            \node at (\mywidth / 2, \myheight*.2125 - .25cm) [white] {\footnotesize \tt
            div.append};
            \node at (\mywidth / 2, -.5) {\tt div.bar};
            
            %Decoration
            \draw [thick, decorate,decoration={brace,amplitude=3pt, raise=3pt}]
            (0,.45*\myheight)
            -- (0,.7*\myheight);
            
        % % %Pointer "Farbe"
            \node (Farbe) at (\mywidth + 1cm, \myheight) {\small Farbe};
            \draw [->] (Farbe.south west) -- (\mywidth - 9,.7*\myheight -9);
            
        % % %Zweiter Balken
            \begin{scope}[xshift = \mywidth + .5cm]
            \setlength{\myheight}{5cm}
            
            %Grundbalken
            \draw [dotted] (0,0) rectangle (\mywidth,\myheight);
            
            %Farbige Balken
            \draw [fill,myblue] (0,0) rectangle (\mywidth,0.75 * \myheight);
            
            %Labels
            \node at (\mywidth / 2, \myheight - .75cm) {\bfseries Text};
            \node at (\mywidth / 2, \myheight*.375 + .25cm) [white] {\bfseries Text};
            \node at (\mywidth / 2, \myheight*.375 - .25cm) [white] {\footnotesize \tt
            div.append};
            \node at (\mywidth / 2, -.5) {\tt div.bar};
            \end{scope}
            
        % % %Dritter Balken
            \begin{scope}[xshift = \mywidth*2 + 1cm]
            \setlength{\myheight}{5.5cm}
            
            %Grundbalken
            \draw [dotted] (0,0) rectangle (\mywidth,\myheight);
            
            %Labels
            \node at (\mywidth / 2, \myheight - .75cm) {\bfseries Text};
            \node at (\mywidth / 2, -.5) {\tt div.bar};
            
            %Decoration
            \draw [thick,decorate,decoration={mirror,brace,amplitude=3pt, raise=3pt}]
            (\mywidth,0)
            -- ++(0,\myheight);
            
        % % %Rechter Text
            \node at (\mywidth*3,3) [left,text width=3cm] {\small
                Höhe $h_b$ von {\tt div.bar} (ist immer relativ zu allen anderen {\tt
                div.bar}) mit:
                \[
                h_b = H \times h_i / h_{max},
                \]
                $H$ ist vorgeschriebene feste Höhe
                $h_i$ eingegebene Höhe und
                $h_{max}$ maximal eingegebene Höhe.
            };
        
        \end{scope}
    \end{tikzpicture}
\end{figure} 
%
\par Das Prinzip sollte folgendes sein: jQuery erstellt beim Klicken auf
Einfügen ein neues \htag{div}, welches entweder eine Klasse \jvar{append} oder
eine Klasse \jvar{bar} erhält. Auf der Seite sollte folgendes entstehen:

\par Es muss also im Stylesheet ein Grundstyle von \jvar{div.bar}, sowie
\jvar{div.append}, Elementen festgelegt werden. Außerdem empfiehlt es sich alle
\jvar{div.bar} in einem anderen Container (z.B. \jvar{div.container}) zu
sammeln.
%
\par Die Aufgabe von jQuery (neben der Erstellung und Platzierung von
Elementen) besteht darin, ihnen die CSS Eigenschaften (z.B. Hintergrundfarbe,
Höhe, ...) mitzugeben und die Einstellungen von bereits existierenden Elementen
zu verändern (z.B. die relative Höhe von \jvar{div.bar} Elementen).